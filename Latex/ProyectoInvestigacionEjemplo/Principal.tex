\documentclass[12pt,letterpaper]{book}
\input{Preambulo.tex}
%https://webdemo.myscript.com/views/math/index.html
%https://www.tablesgenerator.com/
%https://www.mathcha.io/
\begin{document}
\pagenumbering{alph}
\date{noviembre}
\maketitle
\renewcommand{\contentsname}{\centering Glosario}
\tableofcontents
\renewcommand{\listfigurename}{\centering Lista de figuras}
\listoffigures
\renewcommand{\listtablename}{\centering Lista de tablas}
\listoftables
%\fancyhead[ubicacion]{Cadena que aparecera en documento}
\pagestyle{fancy}
\fancyhead[OL]{Universidad Nacional Autonoma de Honduras}%Arriba; Impar: O , Par: E, Centro: C, Derecha: R, Izquierda: L
\fancyfoot[EC]{Cuida el medio ambiente}%Abajo
\input{Contenidos.tex}
%Tablas en LaTeX
\chapter{Creando tablas en modo normal}
\begin{table}[H]
\begin{center}
%\renewcommand{\arratstrech}{1}
\begin{tabular}{l|c|r}
Nombre & Cuenta & Observacion\\\hline\hline
Jose Mauricio & 100101 & JMAR\\
Jose Mauricio & 100101 & JMAR\\
Jose Mauricio & 100101 & JMAR
\end{tabular}
\end{center}
\caption{Esta tabla representa los registros de los estudiantes de la clase B.}
\end{table}
\begin{table}[H]
\begin{tabular}{|l|l|p{5cm}|}
\hline
\rowcolor[HTML]{FFCCC9} 
\textbf{A} & \textbf{B} & \textbf{C}\\ \hline
\rowcolor[HTML]{CBCEFB} 
\textit{1} & \textit{2} & \lipsum[1] \\ \cline{1-2}
\rowcolor[HTML]{CBCEFB} 
\textit{4} & \textit{5} & \textit{6} \\ \hline
\rowcolor[HTML]{CBCEFB} 
\textit{7} & \textit{8} & \textit{9} \\ \hline
\end{tabular}
\end{table}
En este apartado voy a citar a uno de los autores en las bibliografias \cite{Autor}.
%\bibliographystyle{siam}
%\bibliography{Sample.bib}
\begin{thebibliography}{a}
\bibitem[A1]{Autor}
Juan,1998,Como hacer un dibujo.
\end{thebibliography}
\end{document}